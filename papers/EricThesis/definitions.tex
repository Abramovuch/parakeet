%%
%% Place here your \usepackage's. Some recommended packages are already included.
%%

% Graphics:
\usepackage[final]{graphicx}
%\usepackage{graphicx} % use this line instead of the above to suppress graphics in draft copies
%\usepackage{graphpap} % \defines the \graphpaper command

% Indent first line of each section:
\usepackage{indentfirst}

% Good AMS stuff:
%\usepackage{amsthm} % facilities for theorem-like environments
%\usepackage[tbtags]{amsmath} % a lot of good stuff!

% Fonts and symbols:
%\usepackage{amsfonts}
%\usepackage{amssymb}

\usepackage{listings,fixltx2e,lambda,array,times,color,algpseudocode}
\usepackage{stmaryrd}
\usepackage[usenames,dvipsnames]{xcolor}

\definecolor{lightgray}{rgb}{0.96,0.96,0.96}

\lstset{ 
  language=Python,                % the language of the code
  linewidth=1.0\linewidth, 
  xleftmargin=8pt,
  basicstyle=\footnotesize\ttfamily, % Standardschrift
  numbers=none,                   % where to put the line-numbers
  numberstyle=\footnotesize,          % the size of the fonts that are used for the line-numbers
  stepnumber=1,                   % the step between two line-numbers. If it's 1, each line 
                                  % will be numbered
  numbersep=5pt,                  % how far the line-numbers are from the code
  showspaces=false,               % show spaces adding particular underscores
  showstringspaces=false,         % underline spaces within strings
  showtabs=false,                 % show tabs within strings adding particular underscores
  frame=single,                   % adds a frame around the code
  tabsize=2,                      % sets default tabsize to 2 spaces
  captionpos=b,                   % sets the caption-position to bottom
  breaklines=true,                % sets automatic line breaking
  breakatwhitespace=false,        % sets if automatic breaks should only happen at whitespace
  %title=\lstname,                   % show the filename of files included with \lstinputlisting;
                                  % also try caption instead of title
  numberstyle=\tiny\color{gray},        % line number style
  keywordstyle=\color{blue},          % keyword style
  commentstyle=\color{dkgreen},       % comment style
  backgroundcolor=\color{lightgray}, 
  belowskip=0pt, 
  aboveskip=6pt, 
}

% Formatting tools:
%\usepackage{relsize} % relative font size selection, provides commands \textsmalle, \textlarger
%\usepackage{xspace} % gentle spacing in macros, such as \newcommand{\acims}{\textsc{acim}s\xspace}

% Page formatting utility:
%\usepackage{geometry}

%%
%% Place here your \newcommand's and \renewcommand's. Some examples already included.
%%
% \renewcommand{\le}{\leqslant}
% \renewcommand{\ge}{\geqslant}
% \renewcommand{\emptyset}{\ensuremath{\varnothing}}
% \newcommand{\ds}{\displaystyle}
% \newcommand{\R}{\ensuremath{\mathbb{R}}}
% \newcommand{\Q}{\ensuremath{\mathbb{Q}}}
% \newcommand{\Z}{\ensuremath{\mathbb{Z}}}
% \newcommand{\N}{\ensuremath{\mathbb{N}}}
% \newcommand{\T}{\ensuremath{\mathbb{T}}}
% \newcommand{\eps}{\varepsilon}
% \newcommand{\closure}[1]{\ensuremath{\overline{#1}}}
%\newcommand{\acim}{\textsc{acim}\xspace}
%\newcommand{\acims}{\textsc{acim}s\xspace}

%%
%% Place here your \newtheorem's:
%%

%% Some examples commented out below. Create your own or use these...
%%%%%%%%%\swapnumbers % this makes the numbers appear before the statement name.
%\theoremstyle{plain}
%\newtheorem{thm}{Theorem}[chapter]
%\newtheorem{prop}[thm]{Proposition}
%\newtheorem{lemma}[thm]{Lemma}
%\newtheorem{cor}[thm]{Corollary}

%\theoremstyle{definition}
%\newtheorem{define}{Definition}[chapter]

%\theoremstyle{remark}
%\newtheorem*{rmk*}{Remark}
%\newtheorem*{rmks*}{Remarks}

%% This defines the "proo" environment, which is the same as proof, but
%% with "Proof:" instead of "Proof.". I prefer the former.
%\newenvironment{proo}{\begin{proof}[Proof:]}{\end{proof}}
